\documentclass[12pt]{article}
\usepackage{graphicx}
\usepackage{amssymb}

\usepackage[utf8]{inputenc}
\usepackage{multicol}
\usepackage{listings}
\usepackage{amsmath}
\usepackage[left=2cm,right=2cm,top=2cm,bottom=2cm]{geometry}

  
\newlength\tindent
\setlength{\tindent}{\parindent}
\setlength{\parindent}{0pt}
\renewcommand{\indent}{\hspace*{\tindent}}

\DeclareMathOperator*{\argmin}{argmin}
\DeclareMathOperator*{\argmax}{argmax}

\DeclareMathOperator{\tr}{Tr}
\newcommand{\norm}[1]{\left\lVert#1\right\rVert}

\begin{document}

\title{Projet d'imagerie numérique : Inpainting par réseaux \\ \vskip 7px 
\Large Mathis Petrovich et Raphael Bricout
\vskip -2em}
\author{}

\maketitle{}

\section{Introduction}


\section{Comment ça marche ?}

\subsection{Structure du réseau}
\subsection{Entrainement}

\section{Observation sur des exemples}

\subsection{Dataset utilisé pour entrainement : Places2}
ça marche bien

\subsection{Visage (CelebA)}
Marche pas bien du tout car le model n'est pas entrainer dessus.
Ce model n'est pas disponible pour faire des test : TODO github photo

\section{Quesce qu'on peux changer pour observer les couches?}

\subsection{Batchnormlisation}


\subsection{Rajouter du bruit dans les couches}

\subsubsection{Test avec une image de référence}
Test et observation de la dégradation de l'inpainting




\end{document}
