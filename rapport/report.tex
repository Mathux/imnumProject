\documentclass[12pt]{article}
\usepackage{graphicx}
\usepackage{amssymb}

\usepackage[utf8]{inputenc}
\usepackage{multicol}
\usepackage{listings}
\usepackage{amsmath}
\usepackage[left=2cm,right=2cm,top=2cm,bottom=2cm]{geometry}

  
\newlength\tindent
\setlength{\tindent}{\parindent}
\setlength{\parindent}{0pt}
\renewcommand{\indent}{\hspace*{\tindent}}

\DeclareMathOperator*{\argmin}{argmin}
\DeclareMathOperator*{\argmax}{argmax}

\DeclareMathOperator{\tr}{Tr}
\newcommand{\norm}[1]{\left\lVert#1\right\rVert}

\begin{document}

\title{Projet d'imagerie numérique : Inpainting par réseaux \\ \vskip 7px 
\Large Mathis Petrovich et Raphael Bricout
\vskip -2em}
\author{}

\maketitle{}

\section{Introduction}

L'inpainting par réseaux convolutionnels est une application consistant à complèter une image incomplète. Les avancées récentes en deep learning donnent un nouveau souffle à ce processus, en permettant de le traiter d'une manière différente, souvent plus efficace. Cette méthode est encore loin d'être parfaite, mais présente en moyenne de bons résultats, si l'on compare aux méthodes plus traditionnelles. Notre travail consiste ici à analyser les différentes caractéristiques de ce nouveau procédé, à mettre en valeur ses forces et limites, et à essayer de modifier un peu le réseau lui-même pour mieux comprendre son fonctionnement.

\section{Etat de l'art}

L'inpainting n'est pas un problème récent. 

\section{Impact du papier}



\section{Comment ça marche ?}

\subsection{Structure du réseau}
\subsection{Entrainement}

\section{Observation sur des exemples}

\subsection{Dataset utilisé pour entrainement : Places2}
ça marche bien

\subsection{Visage (CelebA)}
Marche pas bien du tout car le model n'est pas entrainer dessus.
Ce model n'est pas disponible pour faire des test : TODO github photo

\section{Qu'est-ce qu'on peut changer pour observer les couches?}

\subsection{Batchnormlisation}


\subsection{Rajouter du bruit dans les couches}

\subsubsection{Test avec une image de référence}
Test et observation de la dégradation de l'inpainting




\end{document}
